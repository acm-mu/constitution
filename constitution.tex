\documentclass{article}
\usepackage[utf8]{inputenc}
\usepackage{enumitem}

\title{The Marquette Chapter of the Association of Computing Machine Constitution}
\date{2021}

\begin{document}

\maketitle

\section*{Preamble}
The intent of this constitution is to declare a Marquette Chapter of ACM, and to outline the purpose, rules, and regulations of this group.

\section{Name}
The name of this organization shall be Association of Computing Machinery - Marquette University, hereby referred to as the Organization.

\section{Purpose}
The purpose of this Organization shall be to act as a group of students and faculty members with an interest in Computing. The Organization will provide a collective resource to all those on campus that would like to learn more about computing topics, but don’t know where to start. It will also allow those who have a passion for computing to teach and inspire others. 

\section{Membership}
\subsection{Regular Membership}
Regular membership shall be open to any full time Marquette University undergraduate student.
\subsection{Associate Membership}
Associate membership shall be open to any part time student, graduate student, professional student, faculty member, staff member, or administrator at Marquette University.
\subsection{Non-discrimination Clause}
Consistent with all applicable federal and state laws and University policies, this Organization and its subordinate bodies and officers shall not discriminate on the basis of race, color, age, sexual orientation, religion, veteran's status, sex, national origin, or disability in its selection of members, educational programs, or activities.
\subsection{Dues}
Dues will not be charged to members of the Organization, but admission may be charged for special events to cover costs.

\section{Officers}
\subsection{Positions}
Officers of the Organization shall be as follows: President, Secretary, External Vice-President, Internal Vice-President, Treasurer, and Webmaster.
\subsection{Election of Officers}
Elections shall be held at the end of every spring semester. Only regular members of the Organization are eligible to hold office and vote in elections. Officers shall be elected by majority vote.
\subsection{Terms}
Officers shall take office immediately following elections and serve for a period of one year.
\subsection{Officer Eligibility}
Officers shall not be on academic or university probation at the time of their elections and throughout their terms of office.
\subsection{Duties of Officers}
\begin{enumerate}[label=(\Alph*)]
\item The President is responsible for conducting Organization business.
\item The External Vice-President responsibilities include organization of external events, maintaining the image of ACM on campus.
\item The Internal Vice-President responsibilities include recruitment of new members, internal social events, and maintains any Organization resources.
\item Checks must be co-signed by the Treasurer and by either the President or a Vice-President.
\item The officer responsible for conducting meetings is the President.
\item The Webmaster is responsible for everything relating to the Organization web site.
\end{enumerate}

\section{Removal of Officers}
\subsection{Removal}
Officers failing to fulfill the given responsibilities and duties may be removed by the regular members of the Organization.
\subsection{Procedures for Removal}
The removal of an officer requires a 2/3 vote of a quorum following the
notification of the officer in question. Such notification shall be provided in
writing no less than seven working days prior to the vote.

\section{Replacement of Officers}
\subsection{President Vacancy}
In the case where the Presidential Office is vacant, the Internal Vice-President will immediately fill the position.
\subsection{Executive Board Vacancy}
All other executive board positions found to be vacant shall be filled by election immediately.

\section{Meetings}
\subsection{Frequency of Meetings}
A regularly scheduled general meeting shall be held at least once a semester. The officers may call additional meetings when the need arises.
\subsection{Quorum}
A quorum shall consist of 50\% of the regular members.
\subsection{Requirement of Quorum}
A quorum shall be present in order for any official business to be conducted.
Official business shall include elections of officers, setting of dues and any other major decisions affecting the Organization.
\subsection{Parliamentary Rules}
Parliamentary Authority shall follow \underline{\textit{Robert’s Rules of Order, Newly Revised}}.

\section{Committees}
\subsection{Committee Creation}
The officers of the Organization shall have the authority to create any
committees, standing or special, that will further the purpose of the
organization.  No committees currently exist for this organization, but if any should be created in the future, this section may be amended to include them.

\section{Affiliation}
This Organization shall be affiliated with the Association of Computing Machinery, and shall abide by its constitution and by-laws in all cases where there is not conflict between their constitution and by-laws and this constitution and/or the rules, regulations, or policies of Marquette University.  In instances of conflict, this constitution and/or rules, regulations or policies of Marquette University shall take precedence over the constitution or by-laws of the Association of Computing Machinery.

\section{Amendments}
\subsection{Notice for Amendments}
All amendments to this constitution require notice of one week prior to being discussed and voted upon.
\subsection{Quorum for Amendments}
All amendments require a 2/3 vote of a quorum for adoption.
\subsection{Amendment Approval}
Amendments become effective only after approval by the Office of Student Development.

\end{document}
